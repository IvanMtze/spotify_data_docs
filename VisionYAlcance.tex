\documentclass[12pt,a4paper]{book}
\usepackage[utf8]{inputenc}
\usepackage[spanish]{babel}
\usepackage{amsmath}
\usepackage{amsfonts}
\usepackage{amssymb}
\usepackage{graphicx}
\usepackage{setspace}
\usepackage[left=2cm,right=2cm,top=2cm,bottom=2cm]{geometry}

\date{\today}

\begin{document}
 	\begin{titlepage}
	\begin{center}
	{\huge \textbf{Universidad Veracruzana}}\\
	\vspace{2cm}  
	{\Large {Visión y Alcance Del Proyecto}}\\
	\vspace{5mm}	
	{\Large {Spotify-Data}}\\
	\begin{figure}[h]
		\centering
		\includegraphics[scale=0.10]{uvlogo}
	\end{figure}
	{\Large {Ingeniería de software}}\\
    \vspace{2cm}
	{\Large {Arenas Cortes Flor Denisse}}\\
	\vspace{5mm}	
	{\Large {García Sosa Oswaldo }}\\
	\vspace{5mm}	
	{\Large {Hernández Morales Gustavo Adolfo }}\\
	\vspace{5mm}	
	{\Large {Martínez Espinosa Gerardo Iván}}\\
	\vspace{5mm}	
	{\Large {Medel Ayohua Víctor Iván}}\\
	\vspace{3mm}	
	{\Large {Ramos García Alberto}}\\
	\vspace{2cm}	
    \rule{8cm}{0.5mm} \\ \Large Vo.bo\\ 
	\end{center}
\end{titlepage}

\tableofcontents
\newpage
\section{Introducción}
\vspace{0.5 cm}
En el presente documento se identifican los aspectos correspondientes a la visión y alcance de este proyecto.\\
\\ El proyecto Spotify-Data es un proyecto donde se requiere poner a prueba conocimientos de administración de proyectos y pruebas de software con respecto al primer parcial.
		
\section{Definiciones}
\vspace{0.5 cm}
\textbf {UML}: Unified Modeling Language, por sus siglas en inglés, la cual traduce
Lenguaje Unificado de Modelado.\\

\textbf {HTML}: HyperText Markup Language, por sus siglas en inglés, es un lenguaje
basado en etiquetas usado en el desarrollo web el cual brinda un estándar para
la definición de la estructura y para la definición de contenido de la página web
como: texto, imágenes y videos.\\

\textbf {Angular}: Framework para desarrollo de aplicaciones web desarrollado en TypeScript, de código abierto y mantenido por Google.\\

\section{Historial de revisiones}
\vspace{0.5 cm}
\begin{table}[h!]
\centering
\begin{tabular}{|p{0.35\linewidth}|p{0.15\linewidth}|p{0.35\linewidth}|p{0.15\linewidth}|}
\hline
\textbf{Nombre}&\textbf{Fecha}&\textbf{Razón del cambio}&\textbf{Versión}
\\\hline
Víctor Iván Medel Ayohua&3/10/2020&Revisión inicial&v1.1\\\hline
Víctor Iván Medel Ayohua&5/10/2020&Definición de la estructura inicial del documento&v1.2\\\hline
Víctor Iván Medel Ayohua&3/10/2020&Redacción de las Definiciones técnicas utilizadas&v1.3\\\hline
Víctor Iván Medel Ayohua&3/10/2020&Redacción de los requerimientos del sistema&v1.4\\\hline
Víctor Iván Medel Ayohua&6/10/2020&Redacción del capitulo 2&v1.5\\\hline
Víctor Iván Medel Ayohua&7/10/2020&Redacción del capitulo 3,4 &v2.1\\\hline
\end{tabular}
\end{table}

\chapter{Requerimientos del negocio}
Los requerimientos del negocio proporcionan la base y la referencia de la necesidad que se pretende satisfacer.\\
A partir de estos requerimientos nosotros identificamos los objetivos y tareas que los usuarios realizaran con esta herramienta.
\section{Escenario}
\vspace{0.5 cm}
Spotify-Data esta destinado a mejorar la experiencia de los usuarios finales y artistas, a través de una pagina principal que permitirá visualizar al usuario final las canciones reproducidas por el en el ultimo mes, a los artistas el numero de reproducciones que han tenido sus canciones en el ultimo mes, el porcentaje que ocupa cada canción del total de reproducciones y los países principales donde sus canciones tienen mas reproducciones.
\section{Oportunidad de negocio}
\vspace{0.5 cm}
El proyecto sera valioso por que mejorara la experiencia de los usuarios y mostrara información detallada de acuerdo a las acciones que ah realizado frecuentemente.
\section{Objetivos del negocio y criterios de éxito}
\vspace{0.5 cm}
Ayudar al usuario mejorando su experiencia de uso en una herramienta de entretenimiento.\\ 
Desde una pagina principal intuitiva donde navegar por las distintas secciones nos muestra la información mas relevante. 
El valor obtenido del proyecto se vera reflejado por el uso cada vez mayor de nuevos usuarios y la aceptación de los mismos.
\newpage
\vspace{0.5 cm}
\section{Necesidades del cliente o del mercado}
\vspace{0.5 cm}
Spotify-data satisface la necesidad de un usuario de internet que quiere conocer estadísticas musicales.\\
Desde las mas populares o mas reproducidas actualmente proporcionando atajos desde su pagina principal con la lista de los sencillos que han consultado en días posteriores.\\

\textbf{Principalmente el sistema deberá cumplir los siguientes requisitos: }
\vspace{0.5 cm}
\begin{itemize}
\item \textit{Se podrá acceder al sistema desde algún navegador.}
\item \textit{El usuario podrá consultar información detallada sobre algún tema musical.}
\item \textit{El usuario tendrá la capacidad de consultar sus playlist.}
\item \textit{El administrador tendrá acceso a una vista con mayor información que la de un usuario final.}
\item \textit{El usuario tendrá la posibilidad de crear una cuenta.}
\item \textit{El usuario tendrá la posibilidad de ver las acciones que ah realizado frecuentemente desde su home del sistema.}
\item \textit{Tendrá la capacidad de refrescar la sección en uso sin tener que refrescar toda la pagina.}
\item \textit{Deberá mostrar fácilmente la información mas relevante a los usuarios y de forma detallada si el lo desea.}
\item \textit{Deberá incorporar caracteres que sean visibles en la mayoría de los navegadores.}
\item \textit{Deberá permitir navegar por las diferentes secciones de interés a los usuarios.}
\item \textit{Deberá mostrar la información agrupada en tablas o graficas.}
\end{itemize}

\chapter{Visión de la solución}

\section{Declaración de la visión}
\vspace{0.5 cm}
Spotify-data es una herramienta de software diseñada para visualizar estadísticas sobre temas musicales con información especifica.\\ 
para poder conocer el numero de reproducciones que tuvo algún tema musical durante una década, saber desde que países estuvo en reproducción ese tema y el porcentaje del total de reproducciones que ocupa ese país.\\ permitiendo que los usuarios conozcan mas acerca de los temas musicales de su preferencia.
\section{Características principales}
\vspace{0.5 cm}
\textbf {Requerimientos funcionales:}
\vspace{0.5 cm}
\begin{itemize}
\item \textit{Se podrá acceder al sistema desde algún navegador.}
\item \textit{Utilizar la base de datos para recopilar la información.}
\item \textit{La información se mostrara en graficas.}
\item \textit{Acceder al sistema a través de un proceso de autentificación.}
\item \textit{El usuario tendrá la capacidad de consultar sus playlist.}
\item \textit{El administrador tendrá acceso a una vista con mayor información que la de un usuario final.}
\item \textit{El usuario tendrá la posibilidad de crear una cuenta.}
\item \textit{El usuario tendrá la posibilidad de ver las acciones que ah realizado frecuentemente desde su home del sistema.}
\item \textit{El administrador podrá visualizar datos estadísticos mas específicos por cada tema musical.}
\item \textit{Deberá permitir navegar por las diferentes secciones de interés a los usuarios.}
\item \textit{El usuario tendrá la opción de ver la información de los temas musicales agrupados en tablas o graficas.}
\item \textit{El usuario no tendrá acceso a la aplicación sin tener una cuenta.}
\end{itemize}
\newpage
\textbf {Requerimientos no funcionales:}
\vspace{0.5 cm}
\begin{itemize}
\item \textit{El tiempo de carga de la información no deberá superar los 30 segundos.}
\item \textit{Presentara una interfaz intuitiva y completa para el fácil manejo de los usuarios.}
\item \textit{Cada pantalla deberá tener un diseño atractivo a la vista del usuario.}
\item \textit{El acceso a los datos deberá ser de forma segura.}
\item \textit{Tendrá la capacidad de conectarse a una api.}
\item \textit{El proyecto sera administrado en git.}
\item \textit{Utilizar kanban automatizados para la administración del repositorio.}
\item \textit{Cada pull request debera ser autorizado por el lider del proyecto.}
\item \textit{Branch protegidos.}
\item \textit{Funcionar en todos los navegadores web modernos.}
\item \textit{Buen manejo de la documentación y control de versiones.}
\end{itemize}

\chapter{Alcance y limitaciones}

\section{Alcance de la version inicial}
\vspace{0.5 cm}
Se ha establecido un lanzamiento inicial que sirva como prueba del concepto final para este primer periodo de evaluación donde se define bien la arquitectura de nuestro sistema.
Esta version inicial consiste en una demostración de los conocimientos adquiridos en las experiencias educativas de administración de proyectos de software y pruebas de software que incluirá lo siguiente: \\
\vspace{0.5 cm}
\begin{itemize}
\item \textit{Diseño completo del sistema.}
\item \textit{Definir las funcionalidades de la vista del usuario final.}
\item \textit{Definir las funcionalidades de la vista del administrador.}
\item \textit{issues definidos para esta primera version.}
\item \textit{Métodos funcionales de los requerimientos establecidos.}
\item \textit{El sistema utilizara Hard-code para esta version.}
\item \textit{Pruebas definidas y programadas para cada componente y modulo.}
\item \textit{Servidor levantado y funcionando correctamente.}
\item \textit{sacar a producción la version inicial inicial del proyecto antes del 9/10/2020}
\item \textit{Sistema en producción con la version inicial.}
\end{itemize} 
\newpage
\section{Alcance de las versiones posteriores}
\vspace{0.5 cm}
Durante los lanzamientos posteriores se concluirá con la funcionalidad total del proyecto.\\
Poniendo en practica los conocimientos adquiridos durante todo el semestre que incluirá lo siguiente: \\
\begin{itemize}
\item \textit{El sistema hará uso de diversas Apis para el funcionamiento del sistema}
\item \textit{Utilizar la base de datos obtenida durante la version inicial.}
\item \textit{El sistema se volverá mas robusto.}
\end{itemize} 
\newpage
\chapter{Contexto del negocio}
\section{Perfil de los involucrados}
Las partes involucradas son personas que participan activamente en un proyecto, que influyen en el resultado del proyecto. Los perfiles de las partes involucradas son:
\begin{table}[h!]
\begin{tabular}{|p{5 cm}|p{5 cm}|p{5 cm}|}
\hline
\textbf{Involucrado}&\textbf{Intereses principales}&\textbf{Limitaciones}
\\\hline

Flor Denisse Arenas Cortes&Definir el diseño de los mockups que definirán y representaran las operaciones del sistema. & Deberán realizarse con herramientas tecnologías especializadas en el área de software.\\\hline

Oswaldo García Sosa&Realizar los issues de desarrollo correspondientes durante cada version. &Cada issue deberá pasar las pruebas correspondientes.\\\hline

Gustavo Hernández Morales&Realizar los issues de desarrollo correspondientes durante cada version &Cada issue deberá pasar las pruebas correspondientes.\\\hline

Gerardo Iván Martínez Espinosa & Administrar, Gestionar y supervisar el desarrollo del proyecto.&Se requiere el uso del modelo git branching e integración continua, con protección a cada branch.\\\hline

Víctor Iván Medel Ayohua &Realizar la Documentación y especificación de los documentos para el desarrollo del proyecto.&Apegarse a las platillas otorgadas por el profesor para la creación de los documentos del proyecto.\\\hline

Luis Alberto Ramos Garcia&Buen de los diagramas de arquitectura que definirán al sistema. & Deberán realizarse los diagramas con herramientas tecnologías especializadas en el área de software.\\\hline
\end{tabular}
\end{table}
\newpage
\section{Prioridades del proyecto}
\vspace{0.5 cm}
\begin{table}[h!]
\begin{tabular}{|p{5 cm}|p{5 cm}|p{5 cm}|}
\hline
\textbf{Prioridad}&\textbf{Objetivo}&\textbf{Rango de tiempo permitido}
\\\hline
Definir el Equipo de trabajo&constituir el equipo de un máximo de 6 integrantes& hasta el 02/10/2020.\\\hline
Planificación&Definir los objetivos del proyecto & del 02/10/2020 05/10/2020.\\\hline
Documentación&La documentación debe ser clara y estar bien organizada& 02/10/2020 - 07/10/2020.\\\hline
Desarrollo&El desarrollo cumpla con los objetivos de la primera version& 02/10/2020 - 08/10/2020.\\\hline
Pruebas&El sistema cumpla con los requerimientos establecidos de la primera version& 02/10/2020 - 08/10/2020.\\\hline
Entregas&Entregar la 1 version el día 9/10/20200 & del 02/10/2020 - 9/10/2020.\\\hline
\end{tabular}
\end{table}

\section{Entorno operativo}
\vspace{0.5 cm}
Pendiente.

  
\end{document}